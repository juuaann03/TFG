\chapter{Introducción}

En la actualidad, los videojuegos se han consolidado como el medio de entretenimiento más influyente del mundo, superando a la industria del cine y la música tanto en impacto cultural como en ingresos económicos \cite{arias2023industria}. Su evolución los ha llevado a convertirse en herramientas con aplicaciones que van mucho más allá del ocio: se utilizan como simuladores en ámbitos militares, en procesos formativos, en entornos de rehabilitación cognitiva, y en la recreación de contextos históricos o culturales \cite{williamson2005video}. Además, cumplen un papel fundamental en la configuración de comunidades globales, donde miles de jugadores conectan diariamente, compartiendo experiencias, conocimientos e intereses comunes.

Historias como la de Mats Steen \cite{van2025profound}, un joven con distrofia muscular cuya vida social se desarrolló casi exclusivamente en mundos virtuales, evidencian el profundo impacto humano de los videojuegos. Este caso pone de manifiesto que los videojuegos no deben entenderse como simples productos digitales, sino como espacios de interacción, expresión y, en muchos casos, inclusión social.

\begin{figure}[H]
	\centering
	\includegraphics[width=1\linewidth]{imagenes/tlou2.jpg}
	\caption[\textbf{Captura de The Last of Us Parte II}.]{\textbf{Captura de The Last of Us Parte II}, elegido Juego del Año 2020 por los principales medios de la industria. Este videojuego alcanza un nivel gráfico y narrativo sobresaliente. \href{https://static1.srcdn.com/wordpress/wp-content/uploads/2024/01/tlou2r-joel-horse.JPG}{https://static1.srcdn.com/wordpress/wp-content/uploads/2024/01/tlou2r-joel-horse.JPG}}
	\label{foto-the-last-of-us-2}
\end{figure}

Hoy en día, gracias a la emulación de consolas clásicas, los videojuegos gratuitos, los títulos \textit{free-to-play}, las ofertas recurrentes y servicios como Game Pass o PS Plus, el acceso al catálogo global de videojuegos es más fácil que nunca. Esta accesibilidad permite a los jugadores explorar experiencias variadas sin incurrir en grandes gastos, lo que ha democratizado el consumo de videojuegos a nivel global.

Sin embargo, esta abundancia también trae consigo un nuevo reto: la sobrecarga de opciones. Aunque la variedad de títulos es inmensa, encontrar un videojuego que se ajuste realmente a nuestras preferencias, circunstancias y contexto personal puede resultar complicado. Esta dificultad se ve acentuada por el hecho de que los lanzamientos de grandes producciones, conocidos como videojuegos \textit{triple A}, requieren cada vez más tiempo, personal y recursos. Casos como \textit{Grand Theft Auto VI}, cuya llegada ha tomado 13 años desde la última entrega numerada, o \textit{The Elder Scrolls VI}, con una espera superior a los 14 años, reflejan una industria donde los ciclos de desarrollo se han alargado significativamente.

Ante esta realidad, los jugadores buscan alternativas: ya sea revisitar títulos anteriores que pasaron por alto, o explorar juegos similares en mecánicas, narrativa o estilo. El problema es que, aunque plataformas como Steam o Xbox ofrecen recomendaciones, estas suelen estar basadas en criterios limitados, como compras recientes o títulos populares, sin tener en cuenta factores importantes como el tipo de dispositivo disponible, las limitaciones económicas, el idioma, la accesibilidad, el tiempo libre o incluso la intención de explorar géneros nuevos.

Para resolver esta problemática, surge la necesidad de una plataforma que entienda al usuario de forma contextual, que sea capaz de adaptar sus recomendaciones no solo a los gustos generales, sino también a sus condiciones puntuales. Es decir, una herramienta que no solo recomiende “lo popular”, sino “lo adecuado”.

Afortunadamente, los avances recientes en inteligencia artificial, en particular en los modelos generativos de lenguaje natural, han abierto una nueva vía para enfrentar este reto. Estos modelos, entrenados con millones de ejemplos —incluyendo información sobre videojuegos, mecánicas, valoraciones de usuarios y contexto narrativo— tienen la capacidad de comprender, procesar y generar texto de forma coherente, ofreciendo respuestas personalizadas y adaptativas en tiempo real.

Si aprovechamos esta tecnología a nuestro favor, podemos construir sistemas inteligentes de recomendación que no solo interpreten nuestras preferencias, sino que aprendan de nuestras decisiones, se integren con nuestras bibliotecas digitales (por ejemplo, mediante el uso de la \textit{Steam API}), y se adapten a nuestros cambios con el tiempo.

El presente proyecto propone precisamente eso: una plataforma que utiliza procesamiento de lenguaje natural y modelos generativos para ofrecer recomendaciones personalizadas de videojuegos. Esta solución permitirá a los usuarios describir sus necesidades en lenguaje natural —como “quiero un juego cooperativo que pueda jugar en español y no requiera conexión permanente”— y obtener una sugerencia adaptada, relevante y precisa.  

Además, la plataforma no se limita a responder consultas, sino que construye un perfil dinámico del usuario, considerando sus gustos, historial, plataforma de juego y otros factores, para mejorar progresivamente las recomendaciones ofrecidas.

A lo largo de este trabajo se explorarán distintas tecnologías relacionadas con la inteligencia artificial, el diseño de interfaces, y los sistemas backend que permiten dar soporte a una solución escalable, funcional e intuitiva. Se documentará el proceso completo de desarrollo, desde los requisitos iniciales hasta las pruebas finales, justificando cada decisión técnica y evaluando el rendimiento del sistema.

\vspace{0.3cm}

El resto del documento se estructura de la siguiente manera:

\begin{itemize}
	\item \textbf{Capítulo 2}: Se explica qué son los modelos generativos de lenguaje y en qué consiste LangChain, además de analizar algunos trabajos previamente realizados.
	
	\item \textbf{Capítulo 3}: Se presentan los objetivos del proyecto, así como la planificación temporal y el presupuesto estimado.
	
	\item \textbf{Capítulo 4}: Se analizan los distintos requisitos de la aplicación, explicando también el proceso seguido para su obtención.
	
	\item \textbf{Capítulo 5}: Se describe la arquitectura general del sistema, detallando los diferentes componentes que lo conforman y su interacción.
	
	\item \textbf{Capítulo 6}: Se explica la implementación de la plataforma, incluyendo las tecnologías utilizadas, el desarrollo de cada módulo y el procedimiento para su ejecución.
	
	\item \textbf{Capítulo 7}: Se recogen las pruebas realizadas para verificar y validar el correcto funcionamiento de la aplicación, incluyendo la reflexión de Antonio Villena Díaz.
	
	\item \textbf{Capítulo 8}: Se exponen las distintas conclusiones obtenidas a lo largo del desarrollo del trabajo, así como posibles líneas futuras de mejora o ampliación.
	
	\item \textbf{Anexo 1}: Explica los requisitos de la aplicación y cómo instalarla y utilizarla correctamente.
\end{itemize}
