
\chapter*{Prefacio}

\begin{center}
{\large\bfseries Sugerencias de juegos con
LangChain: Desarrollo de un sistema web de recomendaciones de
videojuegos personalizadas a través de LangChain}\\
\end{center}
\begin{center}
Juan Manuel Garzón Ferrer\\
\end{center}

%\vspace{0.7cm}
\noindent{\textbf{Palabras clave}: TFG, LangChain, LangGraph, LangSmith, modelos generativos de lenguajes, recomendaciones, sugerencias, videojuegos, recomendación de videojuegos, sugerencias de videojuegos.}\\

{\textbf{Resumen}}\\

Este proyecto se centra en la creación de una plataforma web que ofrezca recomendaciones personalizadas de videojuegos, utilizando la novedosa tecnología Langchain como aspecto clave para orquestar no solo las preferencias del usuario sino también las opiniones de diferentes LLMs como ChatGPT o GEMINI. La plataforma permitirá tanto a usuarios registrados como no registrados interactuar con el sistema, proporcionando recomendaciones basadas en los juegos que han jugado o en nuevas preferencias introducidas en tiempo real. Los usuarios registrados podrán beneficiarse de un historial que mejora las recomendaciones con el tiempo, mientras que los usuarios no registrados recibirán sugerencias inmediatas basadas únicamente en la información proporcionada en la sesión actual y novedades del momento.

La plataforma se integrará con APIs de los diferentes LLMs a través de LangChain para obtener datos extra sobre los videojuegos, como nuevos lanzamientos o clásicos de antaño, popularidad en comunidades específicas, u otros aspectos relevantes como accesibilidad y disponibilidad. El sistema analizará esta información para ofrecer recomendaciones adaptadas a los gustos personales, garantizando que los usuarios descubran nuevos juegos adecuados a sus intereses y necesidades.

Además, el uso de LangChain permite utilizar múltiples LLMs, lo que optimiza el rendimiento del sistema al facilitar la consulta de diferentes fuentes de datos en paralelo. Esto no solo mejora la rapidez de las recomendaciones, sino que también enriquece la calidad de las respuestas al incorporar una variedad de perspectivas y enfoques. LangChain proporciona una arquitectura flexible que se adapta a las necesidades del proyecto, permitiendo escalar y mejorar continuamente las capacidades de la plataforma conforme se integran nuevos datos y se reciben más interacciones de los usuarios.

\newpage
\begin{center}
{\large\bfseries Suggestions for games with LangChain: Development of a web system for personalized video game recommendations using LangChain.}\\
\end{center}
\begin{center}
Juan Manuel Garzón Ferrer\\
\end{center}

%\vspace{0.7cm}
\noindent{\textbf{Keywords}: Bachelor's Thesis, LangChain, LangGraph, LangSmith, large language model, LLMs recommendations, suggestions, video games, video game recommendation, video game suggestions.}\\

\vspace{0.7cm}
\noindent{\textbf{Abstract}}\\

This project focuses on the creation of a web platform that offers personalized video game recommendations, using the innovative LangChain technology as a key aspect to orchestrate not only user preferences but also the opinions of different LLMs such as ChatGPT or GEMINI. The platform will allow both registered and non-registered users to interact with the system, providing recommendations based on the games they have played or new preferences introduced in real-time. Registered users will benefit from a history that improves recommendations over time, while non-registered users will receive immediate suggestions based solely on the information provided during the current session and current trends.

The platform will integrate with APIs from various LLMs via LangChain to collect additional data about video games, such as new releases or classic titles, popularity in specific communities, or other relevant aspects like accessibility and availability. The system will analyze this information to offer recommendations tailored to personal tastes, ensuring that users discover new games suited to their interests and needs.

Furthermore, the use of LangChain allows the utilization of multiple LLMs, optimizing the system's performance by enabling queries from different data sources in parallel. This not only speeds up the recommendations but also enhances the quality of the responses by incorporating a variety of perspectives and approaches. LangChain provides a flexible architecture that adapts to the project's needs, allowing for continuous scaling and improvement of the platform’s capabilities as new data is integrated and more user interactions are received.



\newpage


\noindent\rule[-1ex]{\textwidth}{2pt}\\[4.5ex]

Yo, \textbf{Juan Manuel Garzón Ferrer}, alumno de la titulación TITULACIÓN de la \textbf{Escuela Técnica Superior
de Ingenierías Informática y de Telecomunicación de la Universidad de Granada}, autorizo la
ubicación de la siguiente copia de mi Trabajo Fin de Grado en la biblioteca del centro para que pueda ser
consultada por las personas que lo deseen.

\vspace{6cm}

\noindent Fdo: Juan Manuel Garzón Ferrer

\vspace{2cm}

\begin{flushright}
Granada a 15 de junio de 2025 .
\end{flushright}


\newpage

\noindent\rule[-1ex]{\textwidth}{2pt}\\[4.5ex]

D. \textbf{Ignacio Javier Pérez Gálvez}, Profesor del Departamento de Ciencias de la Computación e Inteligencia Artificial de la Universidad de Granada.



\textbf{Informa:}


Que el presente trabajo, titulado \textit{\textbf{Título del proyecto, Subtítulo del proyecto}}, ha sido realizado bajo su supervisión por \textbf{Juan Manuel Garzón Ferrer}, y autorizamos la defensa de dicho trabajo ante el tribunal que corresponda.

\vspace{0.5cm}

Y para que conste, expiden y firman el presente informe en Granada a 15 de junio de 2025.

\vspace{1cm}

\textbf{Los directores:}

\vspace{1cm}

\noindent \textbf{Ignacio Javier Perez Galvez}

\chapter*{Agradecimientos}

\vspace{1cm}

Poner aquí agradecimientos...


\chapter*{Licencia}

Este Trabajo Fin de Grado está publicado bajo la licencia \textbf{Creative Commons Atribución-NoComercial 4.0 Internacional (CC BY-NC 4.0)}.  
Esto significa que cualquier persona es libre de:

\begin{itemize}
	\item Compartir: copiar y redistribuir el material en cualquier medio o formato.
	\item Adaptar: remezclar, transformar y construir a partir del material.
\end{itemize}

Bajo las siguientes condiciones:

\begin{itemize}
	\item \textbf{Atribución:} Se debe dar crédito adecuado, proporcionar un enlace a la licencia e indicar si se han realizado cambios.
	\item \textbf{No comercial:} No se puede utilizar el material con fines comerciales sin autorización explícita del autor.
\end{itemize}

Más información sobre esta licencia en: \url{https://creativecommons.org/licenses/by-nc/4.0/deed.es}


\begin{center}
	\includegraphics[width=1\textwidth]{imagenes/licencia.png} \\
	\vspace{0.5em}
	\textit{Licencia Creative Commons Atribución-NoComercial 4.0 Internacional (CC BY-NC 4.0)} \\
	\url{https://creativecommons.org/licenses/by-nc/4.0/deed.es}
\end{center}
